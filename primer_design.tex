\documentclass[twocolumn]{article}
%\usepackage{cite}
\usepackage{hyperref}
\hypersetup{colorlinks=true}

\begin{document}
\title{Protocol: Designing DNA-Primers for PCR}
\author{Jack Reddan}

\maketitle{}

%\tableofcontents{}

\section{Purpose}
This protocol is intended for those wanting to design DNA oligomers to be used as primers for PCR.
This protocol will not describe how to identify the sequence you wish to amplify,
since this is variable depending on what you wish to accomplish.
Therefore, a prerequisite for this protocol is to already have a target sequence in mind.

\section{Materials}
\begin{itemize}
	\item Sequence(s) of interest (FASTA format)
	\item Prefered $T_m$ for polymerase
\end{itemize}

\section{Method}
\begin{enumerate}
	\item Identify the first 18--21 base pairs at the 5'-end of the sequence which has a $T_m$ within 1$^\circ$C of the prefered $T_m$.
	\item Ensure that the $\Delta G^\circ$ for this oligonucleotide's homodimer is $\ge -10$ kcal/mol.
	\item It is prefered if the base pair at both the 5'-end and the 3'-end are G/C pairs (a G/C-cap), although, if this is not possible, then prioritize the 3'-end (e.g., 5'-(N)$_n$(G/C)-3' rather than 5'-(G/C)(N)$_n$-3).
	\item Repeat steps 1--3 for the 3'-end of the sequence of interest.
	\item Ensure that the $\Delta G^\circ$ for these oligonucleotides' heterodimer is $\ge -10$ kcal/mol.
	\item Repeat steps 1--5 for each sequence of interest.
\end{enumerate}

\section{Tips and Troubleshooting}
\begin{itemize}
	\item You can use \href{https://benchling.com/editor}{Benchling} for all $T_m$ and $\Delta G^\circ$ calculations.
	\item Attempt to obtain a similar $T_m$ for both primers.
	\item The order of importance for each parameter is as follows:  $\Delta G^\circ$ (homodimerization) $\ge T_m > \Delta G^\circ$ (heterodimerization) $>$ G/C-cap $>$ length. \textbf{Note: All parameters are still important. Only reference this hierarchy if you have tried everything to optimize all parameters.}
	\item All calculations should be done before appending 5'-homology if amplicons are to be used in homology-mediated recombination (e.g., Gibson cloning).
	\item Ensure that concentrations of PCR components are accurate when predicting values for  $T_m$ and $\Delta G^\circ$ (i.e., [Mg$^{2+}$], [primers], [template DNA]).
\end{itemize}

%\bibliography{../../library-hwa_research}
%\bibliographystyle{ieeetr}

\end{document}
